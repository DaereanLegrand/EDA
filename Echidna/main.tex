\documentclass{beamer}
\usepackage[spanish]{babel}
\usepackage{hyperref}
\usepackage{amsmath}
\usepackage{graphicx}
\usepackage{minted}
\usepackage{biblatex}
\usetheme{PaloAlto}

\addbibresource{biblio.bib}

\title{Echidna Clustering}
\author{Frank Roger Salas Ticona}
\date{\today}

\begin{document}

\frame{\titlepage}

\begin{frame}
\frametitle{Clustering Mixed Data}
Existen diversos algoritmos, para agrupar/clusterizar datos uniformes. Y es por ello que existe la necesidad de un algoritmo que nos permita clusterizar datos mixtos o no uniformes.\\
Un ejemplo de datos mixtos, es el \textit{Tráfico en redes}. \\

Tenemos tipos de datos tales como: \textit{SrcIP, DstIP, Protocol, SrcPort, DstPort, bytes}. Podemos observar, que existen tipos de datos jerárquicos como \textit{SrcIP, DstIP}, \textit{bytes} es cuantitativo y \textit{Protocol, SrcPort, DstPort} son de tipo nominal y categóricos.
\end{frame}

\begin{frame}
\frametitle{Distance Functions}
\begin{enumerate}
    \item \textbf{Atributos cuantitativos}:
        Son representados mediante escalares. Sus centroides en el cluster son dados por el promedio de N puntos. Calculamos la distancia entre clusteres usando la distancia Euclidiana.
    \item \textbf{Atributos categóricos o cualitativos}: 
        Un cluster con N puntos es representado por un histograma de frecuencias, de los atributos. De esta manera calculamos la distancia entre clusteres usando la distancia Euclidiana entre la frecuencia de cada atributo.
    \item \textbf{Atributos jerárquicos}: Representado por un árbol. Cada nodo que no es una hoja es la generalización de nodos hoja en el subárbol en dicho nodo.
\end{enumerate}
\end{frame}

\begin{frame}
\frametitle{Distance Functions}
    En un cluster C cuyo atributo jerárquico corresponde a la dirección IP, es representado por un prefijo $\bar{IP}/p$. Calculamos la distancia enter dos clusteres con centroides, $\bar{IP_{1}}/p$ y $\bar{IP_{2}}/p$.
$$d_{h}(C_{1}, C_{2})=32 - p$$
Si $p > 8$ o 24 si $p \leq 8$, $p = CommonPrefix(\bar{IP_{1}}/p / \bar{IP_{2}}/p)$.
\end{frame}

\begin{frame}
\frametitle{Radius Calculation}
Para controlar la varianza de los datos en un cluster, necesitamos alguna medidad del \textit{radio} del cluster. Podemos calcularlo para los atributos cuantitativos y cualitativos, con las desviación de los atributos en el cluster.\\
Para el caso de datos jerárquicos como los IPs, obtenemos \textit{MinIP} y \textit{MaxIP}. Obtenemos un rango tal que $C[i].range=(minIP, maxIP)$, medimos este \textit{radio} con la altura del sub árbol más pequeño en la generalización jerárquica. $$R_{h}=(32-CommonPrefix(minIP, maxIP) / 32)$$.
\end{frame}

\begin{frame}
\frametitle{Cluster formation}
Echidna, es formado por un Árbol CF o CF-Tree \cite{mahmood2006echidna}.\\
Cada cluster como en BIRCH es representado por un vector que contiene estadísticas suficientes para calcular el $centroide$ y $radio$. Cada dato X, comienza desde la raíz siguiendo el camino $P$ hasta un nodo hoja. Se inserta al cluster más cercano y se actualiza en radio. En caso el número de entradas haya alcanzado el tope máximo entonces, el nodo se divide, y hacer el proceso recursivo en hasta llegar al nodo raíz.
\end{frame}

\begin{frame}
\frametitle{Complexity}    
En un árbol balanceado por su altura, con un factor de \textit{branching} B y \textit{m} nodos, se harán $log_{B}m$ comparaciones que on requeridas al momento de realizar una insercion. 
$$O(N * B(1 + log_{B}m))$$
\end{frame}

\begin{frame}
\frametitle{Bibliografía}
\printbibliography
\end{frame}
\end{document} 
