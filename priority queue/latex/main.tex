\documentclass{beamer}
\usepackage[spanish]{babel}
\usepackage{hyperref}
\usepackage{amsmath}
\usepackage{graphicx}
\usepackage{tikz}
\usepackage{minted}

\usetheme{PaloAlto}

\title{Priority Queue}
\author{Frank Roger Salas Ticona}
\date{\today}

\begin{document}

\frame{\titlepage}

\begin{frame}
\frametitle{Definición de Priority Queue}
    Es una estructura de datos similar a las colas y pilas, donde los elementos llevan cierta prioridad asociada, el elemento con la mayor prioridad es el que es utilizado primero.
    \begin{center}
         \begin{tabular}{|c|c|c|}
            \hline
            Priority Queue &  & Queue\\
            \hline
            57 &  & 32 \\
            \hline
            32 &  & 12 \\
            \hline
            19 &  & 57 \\
            \hline
            12 &  & 19 \\
            \hline
        \end{tabular}
    \end{center}
\end{frame}

\begin{frame}
\frametitle{¿Por qué una priority queue?}
Las colas de prioridad son usadas ampliamente en diversos algoritmos tales como el algoritmo de Dijkstra, Huffman, best-first search, entre otros. Lo cual la hace una estructura de datos idónea para resolver distintos tipos de problemas.
\end{frame}

\begin{frame}
\frametitle{¿Cómo es una priority queue?}
Podemos ver a nuestro priority queue como un árbol binário de la siguiente manera:
    \begin{center}
        \begin{tikzpicture}[
roundnode/.style={circle, draw=black!60, minimum size=4mm},
]
            \node[roundnode] {90} 
                child {node[roundnode, xshift = -0.5cm] {80}
                    child {node[roundnode] {10}}
                    child {node[roundnode] {40}}}
                child {node[roundnode, xshift = 0.5cm] {50}
                    child {node[roundnode] {30}}
                    %child {node[roundnode] {30}}
                    }
        \end{tikzpicture}
    \end{center}
Representación en vector:\\
    \begin{center}
         \begin{tabular}{|c|c|c|c|c|c|}
            \hline
            90 & 80 & 50 & 10 & 40 & 30 \\
            \hline
            1 & 2 & 3 & 4 & 5 & 6\\
            \hline
        \end{tabular}
    \end{center}
\end{frame}

\begin{frame}
\frametitle{Top}
Lo principal de un priority queue es obtener el elemento con la mayor prioridad.
    \begin{center}
        \begin{tikzpicture}[
roundnode/.style={circle, draw=black!60, minimum size=4mm},
]
            \node[roundnode, fill=yellow!40] {90} 
                child {node[roundnode, xshift = -0.5cm] {80}
                    child {node[roundnode] {10}}
                    child {node[roundnode] {40}}}
                child {node[roundnode, xshift = 0.5cm] {50}
                    child {node[roundnode] {30}}
                    %child {node[roundnode] {30}}
                    }
        \end{tikzpicture}
    \end{center}
Representación en vector:\\
    \begin{center}
         \begin{tabular}{|c|c|c|c|c|c|}
            \hline
            90 & 80 & 50 & 10 & 40 & 30 \\
            \hline
            1 & 2 & 3 & 4 & 5 & 6\\
            \hline
        \end{tabular}
    \end{center}
\end{frame}

\begin{frame}
\frametitle{Insersión de nuevo elemento}
    \begin{center}
        \begin{tikzpicture}[
roundnode/.style={circle, draw=black!60, minimum size=4mm},
]
            \node[roundnode] {90} 
                child {node[roundnode, xshift = -0.5cm] {80}
                    child {node[roundnode] {10}}
                    child {node[roundnode] {40}}}
                child {node[roundnode, xshift = 0.5cm] {50}
                    child {node[roundnode] {30}}
                    child {node[roundnode, fill=yellow!40] {60}}
                    }
        \end{tikzpicture}
    \end{center}
    \begin{center}
         \begin{tabular}{|c|c|c|c|c|c|c|}
            \hline
             90 & 80 & 50 & 10 & 40 & 30 & 60 \\
            \hline
             1 & 2 & 3 & 4 & 5 & 6 & 7\\
            \hline
        \end{tabular}
    \end{center}
Sin embargo, observamos que no cumple con las condiciones necesarias.
\end{frame}

\begin{frame}
\frametitle{Insersión de nuevo elemento}
Movemos a la posición adecuada: \\
    \begin{center}
        \begin{tikzpicture}[
roundnode/.style={circle, draw=black!60, minimum size=4mm},
]
            \node[roundnode] {90} 
                child {node[roundnode, xshift = -0.5cm] {80}
                    child {node[roundnode] {10}}
                    child {node[roundnode] {40}}}
                child {node[roundnode, xshift = 0.5cm, fill=yellow!40] {60}
                    child {node[roundnode] {30}}
                    child {node[roundnode] {50}}
                    }
        \end{tikzpicture}
    \end{center}
    \begin{center}
         \begin{tabular}{|c|c|c|c|c|c|c|}
            \hline
             90 & 80 & 60 & 10 & 40 & 30 & 50 \\
            \hline
             1 & 2 & 3 & 4 & 5 & 6 & 7\\
            \hline
        \end{tabular}
    \end{center}
\end{frame}

\begin{frame}
\frametitle{Pop o elminicación}
En una priorirty queue se elimina el elemento con la mayor prioridad:
    \begin{center}
        \begin{tikzpicture}[
roundnode/.style={circle, draw=black!60, minimum size=4mm},
]
            \node[roundnode, fill=yellow!40] {90} 
                child {node[roundnode, xshift = -0.5cm] {80}
                    child {node[roundnode] {10}}
                    child {node[roundnode] {40}}}
                child {node[roundnode, xshift = 0.5cm] {60}
                    child {node[roundnode] {30}}
                    child {node[roundnode] {50}}
                    }
        \end{tikzpicture}
    \end{center}
    \begin{center}
         \begin{tabular}{|c|c|c|c|c|c|c|}
            \hline
             90 & 80 & 60 & 10 & 40 & 30 & 50 \\
            \hline
             1 & 2 & 3 & 4 & 5 & 6 & 7\\
            \hline
        \end{tabular}
    \end{center}
\end{frame}

\begin{frame}
\frametitle{Pop o elminicación}
En una priorirty queue se elimina el elemento con la mayor prioridad:
    \begin{center}
        \begin{tikzpicture}[
roundnode/.style={circle, draw=black!60, minimum size=4mm},
]
            \node[roundnode, fill=yellow!40] {50} 
                child {node[roundnode, xshift = -0.5cm] {80}
                    child {node[roundnode] {10}}
                    child {node[roundnode] {40}}}
                child {node[roundnode, xshift = 0.5cm] {60}
                    child {node[roundnode] {30}}
                    }
        \end{tikzpicture}
    \end{center}
    \begin{center}
         \begin{tabular}{|c|c|c|c|c|c|}
            \hline
             50 & 80 & 60 & 10 & 40 & 30 \\
            \hline
             1 & 2 & 3 & 4 & 5 & 6 \\
            \hline
        \end{tabular}
    \end{center}
\end{frame}

\begin{frame}
\frametitle{Pop o elminicación}
En una priorirty queue se elimina el elemento con la mayor prioridad:
    \begin{center}
        \begin{tikzpicture}[
roundnode/.style={circle, draw=black!60, minimum size=4mm},
]
            \node[roundnode] {80} 
                child {node[roundnode, xshift = -0.5cm, fill=yellow!40] {50}
                    child {node[roundnode] {10}}
                    child {node[roundnode] {40}}}
                child {node[roundnode, xshift = 0.5cm] {60}
                    child {node[roundnode] {30}}
                    }
        \end{tikzpicture}
    \end{center}
    \begin{center}
         \begin{tabular}{|c|c|c|c|c|c|}
            \hline
             50 & 80 & 60 & 10 & 40 & 30 \\
            \hline
             1 & 2 & 3 & 4 & 5 & 6 \\
            \hline
        \end{tabular}
    \end{center}
\end{frame}

\begin{frame}
\frametitle{Bibliográfia}
\end{frame}

\end{document} 
